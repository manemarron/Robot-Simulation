\documentclass[10pt]{article}
\usepackage[usenames]{color} %for the color
\usepackage{amssymb} %maths
\usepackage{amsmath} %maths
\usepackage[utf8]{inputenc} %useful to type directly diacritic characters
\usepackage{blindtext}

\begin{document}

\section*{4. Modelado del Sistema}

La din�mica del robot debe ser descrita por un modelo matem�tico para facilitar el desarrollo y eficiencia del sistema de control  de balance angular. Para la implementaci�n f�sica del robot, debemos considerar dos modelos matem�ticos: un p�ndulo invertido y un modelo lineal para un motor DC. El modelo del motor requiere de conocimientos de electr�nica y circuiter�a, por lo que excede los prop�sitos del presente proyecto y nos concentraremos en la formulaci�n te�rica del autobalance y su modelaci�n matem�tica.


\subsection*{4.1. Modelo din�mico para un p�ndulo invertido de dos ruedas}


Primeramente analizaremos el p�ndulo y la din�mica de las ruedas por separado, y al final combinaremos ambos modelos.

\subsubsection*{4.1.1. Ruedas}

Como las ecuaciones de las ruedas son an�logas entre s�, s�lo mostraremos el desarrollo para la rueda derecha.

Por la ley de Newton, la suma de las fuerzas horizontales en \emph{x} es:

\vspace{2 mm}
 $\sum{F_{}x} = Ma$

\vspace{1 mm}
\begin{equation} 
M_{\omega}\ddot{x} = H_{fR} - H_{R}
\end{equation} 
\vspace{2 mm}
La suma de fuerzas alrededor de la rueda nos da:

\vspace{2 mm}
$\sum{M_{o}} = I\alpha$

\vspace{1 mm}
\begin{equation} 
I_{w}\ddot{\theta}_{w} = C_{R} - H_{_{fR}}r
\end{equation} 
\vspace{2 mm}
Acomodando los t�rminos y substituyendo los par�metros del motor DC, la ecuaci�n anterior se convierte en:

\vspace{2 mm}
\begin{equation} 
I_{w}\ddot{\theta}_{w} = \frac{-k_{m}k_{e}}{R}\dot{\theta}_{w} + \frac{k_{m}}{R}V_{a} - H_{_{fR}}r
\end{equation} 
\vspace{2 mm}
\emph{Para la rueda derecha}
\vspace{2 mm}
\begin{equation} 
\Rightarrow M_{w}\ddot{x} = \frac{-k_{m}k_{e}}{Rr}\dot{\theta}_{w} + \frac{k_{m}}{Rr}V_{a} - \frac{I_{w}}{r}\ddot{\theta}_{w} - H_{L}
\end{equation} 
\vspace{2 mm}
\emph{Para la rueda izquierda}
\vspace{2 mm}
\begin{equation} 
\Rightarrow M_{w}\ddot{x} = \frac{-k_{m}k_{e}}{Rr}\dot{\theta}_{w} + \frac{k_{m}}{Rr}V_{a} - \frac{I_{w}}{r}\ddot{\theta}_{w} - H_{R}
\end{equation} 
\vspace{2 mm}
\emph{Tomando en cuenta las transformaciones:}\
\vspace{2 mm}

$\ddot{\theta}_{w}r = \ddot{x} \Rightarrow \ddot{\theta}_{w} = \frac{\ddot{x}}{r}$

\vspace{2 mm}

$\dot{\theta}_{w}r = \dot{x} \Rightarrow \dot{\theta}_{w} = \frac{\dot{x}}{r}$

\vspace{2 mm}
\emph{Y sumando las ecuaciones de las ruedas, obtenemos finalmente:}
\vspace{2 mm}
\begin{equation} 
2(M_{w} + \frac{I_{w}}{r^2})\ddot{x} = \frac{-2k_{m}k_{e}}{Rr^2}\dot{x} + \frac{2k_{m}}{Rr}V_{a} - (H_{L} + H_{R})
\end{equation} 
\vspace{2 mm}

\subsubsection*{4.1.2. P�ndulo Invertido}
A continuaci�n, mostramos c�mo se modela el chassis del robot, lo cual se puede realizar como un p�ndulo invertido.

De nuevo, por la Ley de Newton, la suma de fuerzas \emph{horizontales} es:

\vspace{2 mm}
 $\sum{F_{x}} = M_{p}\ddot{x}$
\vspace{2 mm}
\begin{equation} 
(H_{L} + H_{R}) - M_{p}l\ddot{\theta}_{p} + M_{p}l\dot{\theta}^2_{p}\sin(\theta_{p}) = M_{p}\ddot{x}
\end{equation} 
\vspace{2 mm}

La suma de fuerzas \emph{perpendiculares} al p�ndulo son:

\vspace{2 mm}
 $\sum{F_{xp}} = M_{p}\ddot{x}\cos(\theta_{p})$
\vspace{2 mm}
\begin{equation} 
(H_{L} + H_{R})\cos(\theta_{p}) + (P_{L} + P_{R})\sin(\theta_{p}) - M_{p}g\sin(\theta_{p}) - M_{p}l\ddot{\theta_{p}} = M_{p}\ddot{x}\cos(\theta_{p})
\end{equation} 
\vspace{2 mm}

La suma de \emph{momentos} alrededor del centro de masa del p�ndulo:

\vspace{2 mm}
$\sum{M_{o}} = I\alpha$ , es decir
\vspace{2 mm}
\begin{equation} 
-(H_{L} + H_{R})l\cos(\theta_{p}) - (P_{L} + P_{R})l\sin(\theta_{p}) - (C_{L} + C_{R}) = I_{p}\ddot{\theta_{p}}
\end{equation}  
\vspace{2 mm}

Definiendo el torque aplicado al p�ndulo y despu�s de una transformaci�n lineal tenemos:
\begin{equation} 
C_{L} + C_{R} = \frac{-2k_{m}k_{e}}{R}\frac{\dot{x}}{r} + \frac{2k_{m}}{R}V_{a}
\end{equation}
 
\vspace{2 mm}
Substituyendo en la ecuaci�n (9) anterior y re-ordenando:
\begin{equation}
-(H_{L} + H_{R})l\cos{\theta_{p}} - (P_{L} + P_{R})l\sin{\theta_{p}} - (\frac{-2k_{m}k_{e}}{Rr}\dot{x} + \frac{2k_{m}}{R}V_{a}) = I_{p}\ddot{\theta_{p}}
\end{equation}

\vspace{2 mm}
De esta forma obtenemos las dos siguientes ecuaciones:
\vspace{2 mm}
\begin{equation}
(I_{p} + M_{p}l^2)\ddot{\theta_{p}} - \frac{2k_{m}k_{e}}{Rr}\dot{x} + \frac{2k_{m}}{R}V_{a} + M_{p}gl\sin(\theta_{p}) = -M_{p}l\ddot{x}\cos(\theta_{p})
\end{equation}

\begin{equation}
\frac{2k_{m}}{Rr}V_{a} = (2M_{w} + \frac{2I_{w}}{r^2} + M_{p})\ddot{x} + \frac{2k_{m}k_{e}}{Rr^2}\dot{x} + M_{p}l\ddot{\theta_{p}}\cos(\theta_{p}) - M_{p}I\dot{\theta_{p}^2}\sin(\theta_{p})
\end{equation}

\vspace{2 mm}
\subsection*{4.2 Obtenci�n del Sistema}
\vspace{2 mm}
Linealizando las ecuaciones (12) y (13), asumiendo que $\theta_{p} = \pi + \phi$ , donde $\phi$ representa el �ngulo vertical en direcci�n ascendente y re-ordenando dichas ecuaciones, obtenemos:

\vspace{2 mm}
\begin{equation}
\ddot{\phi} = \frac{M_{p}l}{I_{p} + M_{p}l^2}\ddot{x} + \frac{2k_{m}k_{e}}{Rr(I_{p} + M_{p}l^2)}\dot{x} - \frac{2k_{m}}{R(I_{p} + M_{p}l^2)}V_{a} + \frac{M_{p}gl}{I_{p} + M_{p}l^2}\theta
\end{equation}

\vspace{2 mm}
\begin{equation}
\ddot{x} = \frac{2k_{m}}{Rr(2M_{w} + \frac{2I_{w}}{r^2} + M_{p})}V_{a} - \frac{2k_{m}k_{e}}{Rr^2(2M_{w} + \frac{2I_{w}}{r^2} + M_{p})}\dot{x} + \frac{M_{p}l}{2M_{w} + \frac{2I_{w}}{r^2} + M_{p}}\ddot{\phi}
\end{equation}

\vspace{2 mm}
Despu�s de sustituir las ecuaciones y de una serie de manipulaciones algebr�icas, obtenemos el siguiente \emph{sistema}:
\vspace{2 mm}
\begin{equation}
\  \left( \begin{array}{c}
\dot{x} \\
\ddot{x}\\
\dot{\phi} \\
\ddot{\phi} \end{array} \right) \   =
\  \left( \begin{array}{cccc}
0 & 1 & 0 & 0\\
0 & \frac{2k_{m}k_{2}(M_{p}lr - I_{p} - M_{p}l^2)}{Rr^2\alpha} & \frac{M_{p}^2gl^2}{\alpha} & 0\\
0 & 0 & 0 & 1 \\
0 & \frac{2k_{m}k_{e}(r\beta - M_{p}l)}{Rr^2\alpha} & \frac{M_{p}gl\beta}{\alpha} & 0 \end{array} \right) \ 
\  \left( \begin{array}{c}
x \\
\dot{x}\\
\phi \\
\dot{\phi}\end{array} \right) \ 
+
\  \left( \begin{array}{c}
0 \\
\frac{2k_{m}(I_{p} + M_{p}l^2 - M_{p}lr)}{Rr\alpha}\\
0\\
\frac{2k_{m}(M_{p}l) - r\beta}{Rr\alpha} \end{array} \right) \ V_{a}
\end{equation}

\vspace{2 mm}
donde:
$\beta = (2M_{w} + \frac{2I_{w}}{r^2} + M_{p})$,   
\hspace{4 mm}
$\alpha = (I_{p}\beta + 2M_{p}l^2[M_{w} + \frac{I_{w}}{r^2}])$

\vspace{2 mm}
En este modelo se asume que las ruedas del robot siempre estar�n en contacto con la tierra (suelo) y que no hay "deslizamiento" (slip) de las llantas (por ejemplo, suelo mojado).
 



\end{document}

